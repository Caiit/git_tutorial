\documentclass[12pt,a4paper,oneside]{article}

\usepackage[dutch]{babel}
\usepackage{fancyhdr}

\pagestyle{fancy}
\fancyhead[L]{Rapport projectmatig werken}
\fancyhead[R]{Houda Alberts, 10740287}

\begin{document}
\title{Webprogrammeren en databases \\
Rapport projectmatig werken \\}
\author{Houda Alberts, 10740287 \\
Begeleider: Eva van Weel}
\date{1 februari 2015}
\maketitle
\newpage

\subsection*{\underline{Reflectie op projectplan}}

Het projectresultaat dat we willen hebben, is een goed functionerde discussieforum die er ook simplistisch uitziet. We willen dat je kunt registreren, inloggen, reageren op topics en topics aanmaken. Ook moeten er beheerdsrechten bestaan, zoals het verwijderen en bewerken van topics, gebruikers, berichten en alle andere aspecten van de site. Hierbij hebben we afgesproken dat ik verantwoordelijk zou zijn voor het inlogsysteem en het registratiesysteem. Amor en Mathijs zouden de functionaliteiten van het topic regelen en Fedor wilde zich graag bezighouden met de opmaak van de pagina's en het implementeren van een locatie met behulp van Google Maps. We spraken iedere dag af op Sciencepark om daar te werken aan de site, behalve op vrijdag. We bleven dan ongeveer zeven uur per dag op Sciencepark. Als we bepaalde aspecten niet af kregen, dan maakte we dit in onze eigen tijd af. \\

We zijn niet afgeweken van de contacturen die we maakte, omdat we deze toch nodig hadden.  Deze afspraken zijn voor een groot deel gerealiseerd, maar de implementatie van sommige dingen hadden nog fouten in zich, die uiteindelijk ik eruit heb gehaald. Hierdoor heb ik het gevoel gehad dat ik veel meer heb gedaan dan de rest. Ook omdat het in het begin stroef ging en de andere teamleden vanwege hertentamens de eerste week bijna niks hebben gedaan, heb ik geprobeerd een begin te maken, maar een achterstand kon niet vermeden worden. Hierdoor hebben we bepaalde (extra) functionaliteiten niet kunnen inbouwen zoals een avatar, topics vastzetten/bewerken en berichten bewerken. \\

Gelukkig hebben we richting de tweede en derde week een inhaalslag gemaakt en konden we de basisfunctionaliteiten wel implementeren. Doordat we eerst de functionaliteiten werkend wilde hebben, is de opmaak niet helemaal zoals we hadden gewenst. Over het algemeen ben ik tevreden met het eindresultaat, omdat we dit toch in drie weken hebben gedaan.  

\subsection*{\underline{Zelfinzicht}}

Persoonlijk ben ik niet zo goed in het functioneren in teamverband, omdat ik graag alleen werk. Dit komt voornamelijk omdat ik altijd een doel voor ogen heb, die anderen niet altijd met mij delen. Ook ben ik vrij perfectionistisch, waardoor ik lang op kleine details blijf hangen. In het begin was daarom de communicatie stroef in ons team. Ik nam namelijk te veel hooi op mijn vork en hierdoor kon ik de taken niet goed verdelen. Nadat ik feedback had ontvangen, heb ik het geprobeerd te veranderen. De taken werden toen goed verdeeld en iedereen wist wat hij/zij moest doen. Mijn perfectionisme is naast een zwak ook een sterk punt. Ik zorg er namelijk wel voor dat alles wat gedaan is, ook door anderen, goed is en werkt, maar hierdoor steek ik veel meer tijd in mijn code en dat van de teamleden. Hierdoor kan ik soms boos worden, omdat ik dan veel meer moet doen, dan ik eigenlijk zou willen en dan is de samenwerking ook niet meer zo fijn. \\

Uit het Beblin model kwam ook uit dat ik een willer ben. Uit de test kwam namelijk dat ik een vormer en voorzitter ben. Dit was ook goed te merken, aangezien ik erg pusherig overkwam soms om toch iets gedaan te hebben. Mijn zwakste punt is dat als er iets niet gedaan wordt en ik heb het meerdere malen gevraagd, dan doe ik het zelf. Hierdoor belast ik mezelf en reageer ik boos op anderen.   

\subsection*{\underline{Eigen ontwikkeling}}

Aan de hand van de ontvangen feedback heb ik mijn aanpak veranderd. Ik heb duidelijker geformuleerd wat er gedaan moest worden en door tijdens de gesprekken te vragen of het nog duidelijk was, kon ik het niveau beter inschatten en begreep iedereen elkaar beter. Op deze manier konden taken ook beter verdeeld worden en werd het werken ook prettiger. Wel bleef ik te veel hooi op mijn vork nemen, aangezien sommige taken niet (goed) werden uitgevoerd. Dit kwam ook omdat er niet voor iedereen genoeg tijd was om zijn eigen fouten te vinden en/of te verbeteren. Wel kon ik merken dat de samenwerking veel vloeiender ging nadat de taken beter werden verdeeld en iedereen doelgericht kon werken. Op deze manier hebben we toch een goed eindproduct kunnen leveren. Mijn ontwikkeling is dus positief geweest voor mezelf, maar dus ook voor het team. 

\subsection*{\underline{Gegeven feedback}}

Ik heb geprobeerd om de feedback zo bondig en doelgericht te schrijven. Het doel is ook deels gerealiseerd, aangezien alle teamleden hun gedrag hebben aangepast. De gegeven feedback staat in de bijlage hieronder.

\newpage

\section*{Bijlage:}

\subsection*{Ontvangen feedback:}

Houda heeft veel kennis wat betreft het vak web programmeren, daardoor is Houda heel waardevol voor ons team. Ze heeft een goede werkhouding en neemt vaak zelf het initiatief. Daarnaast levert ze goed werk af en probeert de kennis die zij heeft over te dragen op de overige teamleden. Wel kan Houda soms duidelijker aangeven wat zij van de andere teamleden verwacht, omdat Houda al ervaring heeft kan zij beter inschatten wat er gevraagd wordt bij het maken van een site. \\

Houda heeft al ervaring met het maken van websites. Tevens is zij erg gemotiveerd om een mooi eindproduct te realiseren. Dit maakt haar de meest productieve persoon in de groep. Het gebeurt regelmatig dat bij een meeting blijkt dat Houda thuis al weer een heel stuk van de website gebouwd heeft, zonder dat dit van haar verwacht werd. Dit is natuurlijk erg positief, en duidt op haar motivatie, alhoewel het soms ook zorgt voor overlap. Er werden bijvoorbeeld twee logo’s tegelijkertijd ontworpen, zonder dat wij dat van elkaar wisten. Wellicht neemt zij soms teveel hooi op haar vork en kan  zij sommige zaken misschien beter delegeren aan anderen.
Verder is Houda een goede teamspeler die altijd probeert om zaken zo duidelijk mogelijk uit te leggen aan de rest van de groep. Ze verschijnt ook bijna altijd te vroeg en nooit te laat op meetings. Het is een prettige en leerzame ervaring om met haar samen te werken. \\

Houda neemt de leiding over het technische gedeelte van het project. Ze heeft ook de meeste ervaring hiermee waardoor ze goed weet wat wel en wat niet kan. Hierdoor ziet ze ook problemen van te voren aan komen, wat veel tijd kan besparen. Wel lijkt het dat Houda het niveau van de groepsleden soms verkeerd inschat waardoor haar advies soms van een te hoog of juist te laag niveau is.  

\newpage

\subsection*{Gegeven feedback:}

Amor is een erg goede leider voor de groep. Hij zorgt ervoor dat we vergadermomenten hebben en dat iedereen hier zijn/haar mening kan geven. Hij toont ook veel initiatief door aan te bieden om aantekeningen te maken tijdens onze vergadermomenten. Hij zorgt er ook voor dat iedereen op dat moment een taak krijgt en het er ook mee eens is. Dit maakt de communicatie tussen de groep makkelijker. Wel is het onduidelijk hoe ver Amor is met het beheersen van de kennis die we moeten toepassen. Hierdoor is onduidelijk welke taken hij kan doen die te maken hebben met de implementatie van onze website. \\

Fedor is erg gemotiveerd om de website te maken. Hierdoor wil hij eerst veel informatie tot zich nemen om die vervolgens toe te passen. Dit heeft als gevolg dat hij nog wel bezig is met het leren van de programmeertalen en dat vertraagt de snelheid van het project. Fedor zou nu alvast aan de slag moeten gaan met het toepassen van zijn kennis, zodat hij niet te lang leert en het project sneller gaat. Door zijn motivatie komt Fedor vaak met erg leuke ideeën voor de website, maar het is belangrijk dat eerst de basis van de website er staat, voordat er gekeken wordt naar extensies. Wel probeert Fedor te helpen waar kan tijdens de contacturen als iemand vastloopt en verwoordt hij dit zo goed mogelijk. Dit is erg fijn voor de samenwerking en zorgt voor een fijne sfeer. \\

Mathijs leert erg snel, waardoor hij gelijk zijn kennis toepast op toepassingen op de website. Hierdoor is hij meestal bezig om bepaalde pagina's van de website werkend te krijgen. Wel is hij vrij stil tijdens het werken, waardoor het onduidelijk is of alles goed gaat of niet. Hij doet goed mee met de vergaderingen en probeert ook met nieuwe ideeën te komen als oude ideeën niet werken. Het is ook erg fijn dat hij ook een groot deel van de ideeën uitprobeert, omdat dit een goed beeld geeft wat nou wel en niet kan. Daarnaast kunnen we op deze manier ook goed zien of we het idee aanhouden of toch voor iets anders moeten gaan. 



\end{document}